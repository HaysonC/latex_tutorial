\documentclass[11pt]{article}

\usepackage[margin=1in]{geometry}
\usepackage{hyperref}
\usepackage{xcolor}
\usepackage{tcolorbox}
\usepackage{enumitem}

% Color definitions
\definecolor{overleafBlue}{RGB}{230,245,255}
\definecolor{localGreen}{RGB}{235,250,235}
\definecolor{warningRed}{RGB}{255,240,240}
\definecolor{titleGray}{RGB}{80,80,80}

\tcbset{
    colframe=black,
    boxrule=0.8pt,
    arc=4pt,
    left=8pt,
    right=8pt,
    top=6pt,
    bottom=6pt
}

\begin{document}

\begin{center}
    {\Large \textbf{How to Install and Use LaTeX}}\\
    \vspace{4pt}
    {\large \textbf{Overleaf vs Local Setup}}\\
    \vspace{6pt}
    {\color{titleGray}Last Updated: \today}
\end{center}

\vspace{1em}

\section*{Purpose}
This document defines two approved workflows for writing and compiling LaTeX documents:
\begin{itemize}[nosep]
    \item Directly on Overleaf (cloud-based)
    \item Locally on a personal machine (MacTeX + VSCode)
\end{itemize}

It specifies when to use each option, along with their advantages and limitations.



\section*{Option A: Overleaf (Cloud-Based LaTeX)}

\begin{tcolorbox}[colback=overleafBlue, title=\textbf{Overview}]
Overleaf is a browser-based LaTeX editor that requires no local installation and supports real-time collaboration.
\end{tcolorbox}

\subsection*{Procedure}
\begin{enumerate}[nosep]
    \item Go to \url{https://www.overleaf.com}
    \item Create a new project (blank or template)
    \item Edit \texttt{.tex} files directly in the browser
    \item Click \textbf{Recompile} to generate the PDF
\end{enumerate}

\subsection*{Pros}
\begin{itemize}[nosep]
    \item Very easy to use (no setup required)
    \item Real-time collaboration and commenting
    \item Automatic package management
    \item Ideal for quick drafts and shared documents
\end{itemize}

\subsection*{Cons}
\begin{itemize}[nosep]
    \item Compile time limits (especially on free tier)
    \item Can become slow or unstable near major conference deadlines
    \item Limited control over compiler versions and system tools
    \item Less suitable for large projects with many figures or custom scripts
\end{itemize}

\begin{tcolorbox}[colback=warningRed, title=\textbf{Operational Risk}]
During peak academic deadlines (e.g., conference submissions), Overleaf servers may experience latency or failed compiles.
\end{tcolorbox}

\section*{Option B: Local LaTeX (MacTeX + VSCode)}

\begin{tcolorbox}[colback=localGreen, title=\textbf{Overview}]
Local compilation uses a full LaTeX distribution installed on your machine, typically with VSCode and LaTeX Workshop.
\end{tcolorbox}

\subsection*{Procedure}
\begin{enumerate}[nosep]
    \item Install MacTeX from \url{https://www.tug.org/mactex/}
    \item Install VSCode and the \textbf{LaTeX Workshop} extension
    \item Open the project folder locally
    \item Compile using \texttt{latexmk}, \texttt{pdflatex}, or \texttt{xelatex}
\end{enumerate}

\subsection*{Local Installation (macOS) --- Command Line Alternatives}
Alternatively, if you prefer not to install via the graphical \texttt{.dmg} installer, the following command-line approaches are common.

\begin{tcolorbox}[colback=localGreen, title=\textbf{Option 1: Install via Homebrew (recommended for CLI users)}]
	\textbf{MacTeX (full distribution, large download):}
\begin{verbatim}
brew update
brew install --cask mactex

# Verify
latex --version
pdflatex --version
latexmk -v
\end{verbatim}

\textbf{MacTeX no-GUI variant (still TeX Live, smaller than full MacTeX):}
\begin{verbatim}
brew update
brew install --cask mactex-no-gui
\end{verbatim}

\textbf{BasicTeX (minimal TeX Live; install packages as needed):}
\begin{verbatim}
brew update
brew install --cask basictex

# BasicTeX usually requires installing missing LaTeX packages via tlmgr
sudo tlmgr update --self
\end{verbatim}
\end{tcolorbox}

\begin{tcolorbox}[colback=localGreen, title=\textbf{Option 2: Install from a downloaded installer using \texttt{installer}}]
This is useful for offline installs or when you already have the installer locally.

	\textbf{If you have a \texttt{.pkg} file:}
\begin{verbatim}
# Example: install a local PKG
sudo installer -pkg /path/to/MacTeX.pkg -target /
\end{verbatim}

	
\textbf{If you have a \texttt{.dmg} but want a fully CLI flow (no GUI):}
\begin{verbatim}
# Mount the DMG
hdiutil attach /path/to/MacTeX-*.dmg

# Find the PKG inside the mounted volume
ls /Volumes
ls "/Volumes/MacTeX-*/"

# Install the PKG (path may vary by release)
sudo installer -pkg "/Volumes/MacTeX-*/MacTeX.pkg" -target /

# Unmount when done
hdiutil detach "/Volumes/MacTeX-*/"
\end{verbatim}
\end{tcolorbox}

\subsection*{Post-Install Checks (macOS)}
\begin{enumerate}[nosep]
    \item Confirm TeX binaries are on your PATH:
    \begin{verbatim}
which pdflatex
which latexmk
\end{verbatim}
    \item If commands are not found, macOS typically uses \texttt{/Library/TeX/texbin}. You can add it to your shell configuration (\texttt{~/.zshrc}):
    \begin{verbatim}
export PATH="/Library/TeX/texbin:$PATH"
\end{verbatim}
    \item (Optional) Update TeX Live packages (may take time and requires admin privileges in some setups):
    \begin{verbatim}
sudo tlmgr update --self
sudo tlmgr update --all
\end{verbatim}
\end{enumerate}

\subsection*{Pros}
\begin{itemize}[nosep]
    \item Highly reliable and robust
    \item No compile time or project size limits
    \item Full control over compiler, packages, and build process
    \item Suitable for large documents, heavy figures, and automation
\end{itemize}

\subsection*{Cons}
\begin{itemize}[nosep]
    \item Initial setup is more complicated
    \item Debugging LaTeX errors requires more familiarity
    \item Requires manual package management
\end{itemize}

\begin{tcolorbox}[colback=localGreen!85!white, title=\textbf{Best Practice}]
Local compilation is strongly recommended for final submissions, large projects, and deadline-critical work.
\end{tcolorbox}


\section*{Recommendation Summary}

\begin{tcolorbox}[colback=white]
\begin{itemize}
    \item Use \textbf{Overleaf} for early drafts, collaboration, and lightweight documents.
    \item Use \textbf{Local LaTeX} for serious writing, large projects, and final submissions.
\end{itemize}
\end{tcolorbox}


\end{document}
